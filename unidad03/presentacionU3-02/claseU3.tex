\documentclass{beamer}

\usetheme[progressbar=foot,numbering=fraction]{metropolis}



\usepackage{appendixnumberbeamer}


%centrogeo orange
\definecolor{beamer@centrogeo}{RGB}{219,132,61}

%rareblue
\definecolor{beamer@centrogeo}{RGB}{2, 104, 161}

%itq orange
\definecolor{beamer@centrogeo}{RGB}{239, 158, 9}

% centrogeo brown
\definecolor{color2}  {RGB}{110,60,20}

\definecolor{color2}  {RGB}{61, 88, 122}

%centrogeo light orange
\definecolor{beamer@centrogeoLightOrange} {RGB}{255,172,101}% 

\definecolor{beamer@centrogeoLightOrange} {RGB}{61, 104, 161}% 

%https://www.centrogeo.org.mx/lgac-ciencias-de-informacion-geoespacial/lgac-cig-ciencias-de-informacion-geoespacial


\setbeamercolor{normal text}{fg=black,bg=white}
\setbeamercolor{alerted text}{fg=beamer@centrogeoLightOrange}
\setbeamercolor{example text}{fg=green!50!black}

\setbeamercolor{structure}{fg=color2}

\setbeamercolor{background canvas}{parent=normal text}
\setbeamercolor{background}{parent=background canvas}
	
\setbeamercolor{palette primary}{fg=white,bg=beamer@centrogeo} % changed this
\setbeamercolor{palette secondary}{use=structure,fg=structure.fg!100!green} % changed this
\setbeamercolor{palette tertiary}{use=structure,fg=structure.fg!100!green}



\usepackage[spanish, english]{babel}
\usepackage[utf8]{inputenc}

\usepackage{booktabs}
\usepackage[scale=2]{ccicons}
\usepackage{url}
%\usepackage{subfig}
\usepackage{subcaption}

\usepackage{caption}




%\usepackage{dsmath}

\usepackage{pgfplots}
\usepgfplotslibrary{dateplot}

\usepackage{xspace}
\newcommand{\themename}{\textbf{\textsc{metropolis}}\xspace}

\usepackage{algorithm}% http://ctan.org/pkg/algorithms
\usepackage[noend]{algpseudocode}

\usepackage{graphicx}
%\usepackage{subcaption}
\usepackage{bibentry}
\usepackage{verbatim}
\usepackage{fancyvrb}
\usepackage{moreverb}

\theoremstyle{definition}
\newtheorem{defn}{Definici\'{o}n}[section]
\newtheorem{conj}{Conjecture}[section]
\newtheorem{exmp}{Example}[section]



\titlegraphic{
%\includegraphics[height=1.0cm]{figures/logo_cicige_L.png} \hfill  

\includegraphics[height=1.5cm]{TecNM_logo.png}  
\hfill 
\includegraphics[height=1.5cm]{itq.png} \hfill 
%5\includegraphics[height=0.9cm]{figures/EscudoDeLaUAQ.jpg}

}

\title{Internet de las Cosas:  Unidad 3}
\subtitle{Sensores y actuadores}


\author{Dr. Rodrigo López Farías. \\}

\institute{Instituto Tecnológico de Querétaro \\ Ingeniería en Sistemas Computacionales}

\date{Ciclo escolar Ene-Junio 2020}


\begin{document}

\frame{\maketitle}


\begin{frame}{Índice}
  \setbeamertemplate{section in toc}[sections numbered]
  \tableofcontents[hideallsubsections]
\end{frame}




\section{Proyecto Integrador Unidad 5}
\begin{frame}{5.1 Diseño de sistemas empleando plataformas de IoT, incluyendo documentacion y diagramacion.}

Propuesta de arquitectura general: Conectar sensores a la Placa de desarrollo (Arduino Uno con ESP8266 o NODMCU) y transmitir por medio de mqtt a un PC ( RaspberryPi).

Aplicación propuesta: Orientar esta arquitectura a la construcción de un dispositivo portable (Bicicleta - Caminando) para medición de gases contaminantes a nivel de suelo, utilizando un GPS, acelerómetro-giroscopio, sensor de gases contaminantes.
\end{frame}

\section{Ante-proyecto propuesto (5.1) 1/2}

\begin{frame}{Pre-propuestas?}
Equipo 1 :  Luis Angel - Bogar

Equipo 2 :  Ariadna - Carlos

\end{frame}

\section{Microcontroladores e IDC}

\begin{frame}
Ya revisamos:
\begin{itemize}
\item Un protocolo de mensajes para IDC y como enviar y recibir mensajes para controlar ejecución de dispositivos virtuales (Conectividad por internet). 
\item Manejo general de entradas y salidas digitales y analógicas de GPIO en una tarjeta de desarrollo. (Programación de objetos) 
\item Falta: Unir Tarjeta de desarrollo + MQTT 
\end{itemize}


\end{frame}

\begin{frame}{Internet de las cosas con ESP8266-01}

\begin{itemize}
\item Opción 1: Arduino - ESP8266. 
	\begin{itemize}
	\item Mas costoso (precio de arduino uno  + esp8266) \url{https://bit.ly/2R2MQnn}
	\item Mas complicado de programar (alambrar arduino uno al esp8266-01) 
	\item La comunicación de los sensores con el ESP es através de mensajes por puerto serial.
	\end{itemize}
\item Opcion dos: NODMCU. 
	\begin{itemize}
		\item Mas barato y mas prestaciones. (cuesta un poco mas caro que el esp8266)
		\item Más fácil de programar (ESP8266 ya está microalambrado al microcontrolador) 
		\item compatible con arduino.
		\item https://bit.ly/3azPV5X. (Les dejo de tarea que lo revisen)
	\end{itemize}
  
\end{itemize}

\end{frame}

\begin{frame}{Ejemplo comunicación del ESP8266 con MQTT de NODMCU }

Configuración
\begin{itemize}
\item Instalar esp8266 NODEMCU 0.9. https://bit.ly/2UTrCZY
\item Instalar pubsub desde el manejador de librerias
\item Instrucciones adicionales estan en el ejemplo main_mqtt
\end{itemize}

\end{frame}














\end{document}